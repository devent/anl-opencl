% Copyright (C) 2021 Erwin Müller <erwin@muellerpublic.de>
% Released as open-source under the Apache License, Version 2.0.
%
% ****************************************************************************
% ANL-OpenCL :: Bundle POM
% ****************************************************************************
%
% Copyright (C) 2021 Erwin Müller <erwin@muellerpublic.de>
%
% Licensed under the Apache License, Version 2.0 (the "License");
% you may not use this file except in compliance with the License.
% You may obtain a copy of the License at
%
%      http://www.apache.org/licenses/LICENSE-2.0
%
% Unless required by applicable law or agreed to in writing, software
% distributed under the License is distributed on an "AS IS" BASIS,
% WITHOUT WARRANTIES OR CONDITIONS OF ANY KIND, either express or implied.
% See the License for the specific language governing permissions and
% limitations under the License.
%
% ****************************************************************************
% ANL-OpenCL :: Bundle POM is a derivative work based on Josua Tippetts' C++ library:
% http://accidentalnoise.sourceforge.net/index.html
% ****************************************************************************
%
% Copyright (C) 2011 Joshua Tippetts
%
%   This software is provided 'as-is', without any express or implied
%   warranty.  In no event will the authors be held liable for any damages
%   arising from the use of this software.
%
%   Permission is granted to anyone to use this software for any purpose,
%   including commercial applications, and to alter it and redistribute it
%   freely, subject to the following restrictions:
%
%   1. The origin of this software must not be misrepresented; you must not
%      claim that you wrote the original software. If you use this software
%      in a product, an acknowledgment in the product documentation would be
%      appreciated but is not required.
%   2. Altered source versions must be plainly marked as such, and must not be
%      misrepresented as being the original software.
%   3. This notice may not be removed or altered from any source distribution.
%
%
% ****************************************************************************
% ANL-OpenCL :: Bundle POM bundles and uses the RandomCL library:
% https://github.com/bstatcomp/RandomCL
% ****************************************************************************
%
% BSD 3-Clause License
%
% Copyright (c) 2018, Tadej Ciglarič, Erik Štrumbelj, Rok Češnovar. All rights reserved.
%
% Redistribution and use in source and binary forms, with or without modification, are permitted provided that the following conditions are met:
%
% * Redistributions of source code must retain the above copyright notice, this list of conditions and the following disclaimer.
%
% * Redistributions in binary form must reproduce the above copyright notice, this list of conditions and the following disclaimer in the documentation and/or other materials provided with the distribution.
%
% * Neither the name of the copyright holder nor the names of its contributors may be used to endorse or promote products derived from this software without specific prior written permission.
%
% THIS SOFTWARE IS PROVIDED BY THE COPYRIGHT HOLDERS AND CONTRIBUTORS "AS IS" AND ANY EXPRESS OR IMPLIED WARRANTIES, INCLUDING, BUT NOT LIMITED TO, THE IMPLIED WARRANTIES OF MERCHANTABILITY AND FITNESS FOR A PARTICULAR PURPOSE ARE DISCLAIMED. IN NO EVENT SHALL THE COPYRIGHT HOLDER OR CONTRIBUTORS BE LIABLE FOR ANY DIRECT, INDIRECT, INCIDENTAL, SPECIAL, EXEMPLARY, OR CONSEQUENTIAL DAMAGES (INCLUDING, BUT NOT LIMITED TO, PROCUREMENT OF SUBSTITUTE GOODS OR SERVICES; LOSS OF USE, DATA, OR PROFITS; OR BUSINESS INTERRUPTION) HOWEVER CAUSED AND ON ANY THEORY OF LIABILITY, WHETHER IN CONTRACT, STRICT LIABILITY, OR TORT (INCLUDING NEGLIGENCE OR OTHERWISE) ARISING IN ANY WAY OUT OF THE USE OF THIS SOFTWARE, EVEN IF ADVISED OF THE POSSIBILITY OF SUCH DAMAGE.

\subsection{OpenCL Kernel}

The basic structure of a kernel is give as an example in listings \ref{lst:kernel_example}.
The kernel expects two parameters to generale an image as a texture. The first
are the \ANLOpenCLTypeIndex{SMappingRanges} that contains the parameters of the mapping
as entered by the user on the Mapping \ref{sec:mapping} window. The second parameter
is the \ANLOpenCLTypeIndex{image2d\_t} image output.

The kernel code can contain variables that are inserted before the code is build.
One of those variables is \ANLOpenCLTypeIndex{\$insert\_localMapRange}. This variable
is replaced with the code in listings \ref{lst:insert_local_map_range}.
\ANLOpenCLType{\$insert\_localMapRange} contains code that divides the coordinates
space given in the mapping ranges into small peaces and allows the noise functions
to operate in the local group on the divided part. After this code was inserted 
the \code{const int i} variable contains the current index in the \code{vector3 coord}
variable, i.e. reading the coordinate from the position \code{coord[i]} will return
the correct coordinate in the local group. Additionally, the variable
\code{struct SMappingRanges ranges} will contain the local mapping ranges.

The variable \ANLOpenCLTypeIndex{\$localSize} contains the size of the part of the
image that we want to process in pixels. Currently it is set to a maximum size of 32.
That means that an image of the size 1024x1024 is divided into 32x32 ($1024/32=32$) parts
with the size of 32x32 pixels. Each noise function will be called in the local group
only on the 32x32 pixels part.

The variable \ANLOpenCLTypeIndex{\$z} contains the Z value from the Image \ref{sec:image} section.

\begin{lstlisting}[caption={Kernel Example},label={lst:kernel_example},language=OpenCL]
#include <opencl_utils.h>
#include <noise_gen.h>
#include <imaging.h>
#include <kernel.h>

kernel void map2d_image(
global struct SMappingRanges *g_ranges,
write_only image2d_t output
) {
    $insert_localMapRange
    const float a = 0.5;
    long seed = 200;
    const float r = value_noise3D(coord[i], seed, linearInterp);
    const float g = value_noise3D(coord[i], seed*2, linearInterp);
    const float b = value_noise3D(coord[i], seed*2, linearInterp);
    const float f = simpleFractalLayer3(coord[i], 
value_noise3D, seed*10, linearInterp, 10, 1, 
false, 0, 1, 0, 0);
    write_imagef(output, (int2)(g0, g1), (float4)(r*f, g*f, b*f, a));
}
\end{lstlisting}

\begin{lstlisting}[caption={Local Map Range Code},label={lst:insert_local_map_range},language=OpenCL]
const size_t g0 = get_global_id(0);
const size_t g1 = get_global_id(1);
const size_t w = get_global_size(0);
const size_t h = get_global_size(1);
const size_t l0 = get_local_id(0);
const size_t l1 = get_local_id(1);
const size_t lw = get_local_size(0);
const size_t lh = get_local_size(1);
local vector3 coord[$localSize * $localSize];
local struct SMappingRanges ranges;
if (l0 == 0 && l1 == 0) {
    const REAL sw = (g_ranges->mapx1 - g_ranges->mapx0) / w;
    const REAL sh = (g_ranges->mapy1 - g_ranges->mapy0) / h;
    const REAL x0 = g_ranges->mapx0 + g0 * sw;
    const REAL x1 = g_ranges->mapx0 + g0 * sw + sw * lw;
    const REAL y0 = g_ranges->mapy0 + g1 * sh;
    const REAL y1 = g_ranges->mapy0 + g1 * sh + sh * lh;
    set_ranges_map2D(&ranges, x0, x1, y0, y1);
    map2D(coord, calc_seamless_none, ranges, lw, lh, $z);
}
work_group_barrier(CLK_LOCAL_MEM_FENCE);
const int i = (l0 + l1 * lh);
\end{lstlisting}
