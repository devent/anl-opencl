\subsection{OpenCL Kernel}

The basic structure of a kernel is give as an example in listings \ref{lst:kernel_example}.
The kernel expects two parameters to generale an image as a texture. The first
are the \ANLOpenCLTypeIndex{SMappingRanges} that contains the parameters of the mapping
as entered by the user on the Mapping \ref{sec:mapping} window. The second parameter
is the \ANLOpenCLTypeIndex{image2d\_t} image output.

The kernel code can contain variables that are inserted before the code is build.
One of those variables is \ANLOpenCLTypeIndex{\$insert\_localMapRange}. This variable
is replaced with the code in listings \ref{lst:insert_local_map_range}.
\ANLOpenCLType{\$insert\_localMapRange} contains code that divides the coordinates
space given in the mapping ranges into small peaces and allows the noise functions
to operate in the local group on the divided part. After this code was inserted 
the \code{const int i} variable contains the current index in the \code{vector3 coord}
variable, i.e. reading the coordinate from the position \code{coord[i]} will return
the correct coordinate in the local group. Additionally, the variable
\code{struct SMappingRanges ranges} will contain the local mapping ranges.

The variable \ANLOpenCLTypeIndex{\$localSize} contains the size of the part of the
image that we want to process in pixels. Currently it is set to a maximum size of 32.
That means that an image of the size 1024x1024 is divided into 32x32 ($1024/32=32$) parts
with the size of 32x32 pixels. Each noise function will be called in the local group
only on the 32x32 pixels part.

The variable \ANLOpenCLTypeIndex{\$z} contains the Z value from the Image \ref{sec:image} section.

\begin{lstlisting}[caption={Kernel Example},label={lst:kernel_example},language=OpenCL]
#include <opencl_utils.h>
#include <noise_gen.h>
#include <imaging.h>
#include <kernel.h>

kernel void map2d_image(
global struct SMappingRanges *g_ranges,
write_only image2d_t output
) {
    $insert_localMapRange
    const float a = 0.5;
    long seed = 200;
    const float r = value_noise3D(coord[i], seed, linearInterp);
    const float g = value_noise3D(coord[i], seed*2, linearInterp);
    const float b = value_noise3D(coord[i], seed*2, linearInterp);
    const float f = simpleFractalLayer3(coord[i], 
value_noise3D, seed*10, linearInterp, 10, 1, 
false, 0, 1, 0, 0);
    write_imagef(output, (int2)(g0, g1), (float4)(r*f, g*f, b*f, a));
}
\end{lstlisting}

\begin{lstlisting}[caption={Kernel Example},label={lst:insert_local_map_range},language=OpenCL]
const size_t g0 = get_global_id(0);
const size_t g1 = get_global_id(1);
const size_t w = get_global_size(0);
const size_t h = get_global_size(1);
const size_t l0 = get_local_id(0);
const size_t l1 = get_local_id(1);
const size_t lw = get_local_size(0);
const size_t lh = get_local_size(1);
local vector3 coord[$localSize * $localSize];
local struct SMappingRanges ranges;
if (l0 == 0 && l1 == 0) {
    const REAL sw = (g_ranges->mapx1 - g_ranges->mapx0) / w;
    const REAL sh = (g_ranges->mapy1 - g_ranges->mapy0) / h;
    const REAL x0 = g_ranges->mapx0 + g0 * sw;
    const REAL x1 = g_ranges->mapx0 + g0 * sw + sw * lw;
    const REAL y0 = g_ranges->mapy0 + g1 * sh;
    const REAL y1 = g_ranges->mapy0 + g1 * sh + sh * lh;
    set_ranges_map2D(&ranges, x0, x1, y0, y1);
    map2D(coord, calc_seamless_none, ranges, lw, lh, $z);
}
work_group_barrier(CLK_LOCAL_MEM_FENCE);
const int i = (l0 + l1 * lh);
\end{lstlisting}
